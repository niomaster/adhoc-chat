\documentclass{article}
\usepackage[utf8]{inputenc}

\title{WiFi Ad-Hoc Chat Application}
\author{Group 1}
\date{April 2014}

\begin{document}

\maketitle

\tableofcontents

\section{Introduction}
For this project we had to design and implement a WiFi-based multi-hop chat application between 4 wireless devices. A normal chat application uses a server to manage the connections and to route data packets between clients. With a multi-hop adhoc chat  application there is no server to manage these connections nor to route data packets. Each device should broadcast its presence periodically so that it is known which devices are in the network. UDP is used to communicate and therefore link unreliability and packet lost probability should be taken into account. Thus the challenge of the project is to make a serverless connection and to make a protocol that enables the application to work correctly with packets in the correct order and no packet loss since this is essential for a chat application.

\section{Description of System Design}

\section{Required Components}

\section{Design Choices}

\section{Test cases} (Max 3 pages)
\subsection{} //Make subsection for each test case
//Questions to be answered in each subsection:
// Which behavior do you expect of the system?
// What is the exact test setup? (make sure that tests can be repeated based on description)
// What does the test scenario look like?
// What are the exact steps you should take during the tests?
// What performance metrics do you consider for the system?
// What is the overall performance of the system?

\section{Performance Evaluation}

\section{Planning}
    \subsection{Routing Algorithm}
    The routing algorithm makes sure that all packets are delivered to the correct client. This includes packet loss, packet order delivery, design and implementation. \\
    It will we made by: Pieter \& Frans. \\
    It should be finished on: 10th of April 2014.

    \subsection{Chat Application}
    Chat application consists of the Graphics User Interface and it should implement the protocol. It therefore includes the design and implementation. \\
    It will be made by: Laurens \& Sophie. \\
    It should be finished on: 10th of April 2014.

    \subsection{Encryption}
    This can be added when we have enough time and the previous named functions already work. \\
    It might be made by Pieter \& Frans. \\
    It should be finished on: 15th of April 2014.

    \subsection{Report}
    Writing of the report. \\
    It will be made by: Sophie \& Laurens. \\
    It should be finished on: 15th of April 2014.

    \subsection{Test Plan}
    Writing of a test plan. \\
    It will be made by Laurens \& Sophie. \\
    It should be finished on: 15th of April 2014.

    \subsection{Demonstration}
    Unclear as of now. \\
    It should be finished on: 16th of April 2014.

    \subsection{Timeline}

    |7th||||10th||||||||15th|16th| \\
    We start on the 7th of April, we made a planning for the rest of the project. We tried to have something minimal on the 10th of April when we had to give a demo and we could add extra components as soon as we had a minimal working version.
    We were unable to do this. We had the components but the connection between these components did not work yet.
    Pieter and Frans should have finished the routing algorithm and Laurens and Sophie should have finished the chat application.
    On the 15th we tried to have something working with extras (such as emoticons and encryption) and we tried to finish the test plan.


\end{document}
